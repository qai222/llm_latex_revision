%! Author = alexq
%! Date = 11/2/2024

% Preamble
\documentclass[11pt]{article}

% Packages
\usepackage{amsmath}

% Document
\begin{document}

    \section{Introduction}

    Data-driven methods are now routinely employed in the physical sciences. A trend toward the use of supervised machine learning (ML) techniques has increased the need for structured data, i.e., data represented using a standardized data schema. In most scientific communities, however, data is stored and communicated predominantly via unstructured documents and prose, with only a few exceptions.\cite{mercado_data_2023} Synthetic organic chemistry is no exception. Reaction procedures and details are commonly recorded as free text in journal publications, patents, or electronic lab notebooks (ELNs). Manual information extraction and curation are still widely used to construct structured datasets from unstructured texts.\cite{gabrielson_scifinder_2018, lawson_making_2014} An automated method to extract structured reaction data from unstructured texts would accelerate efforts to use historical reaction data for data-driven discovery.

    As an information extraction task, structured data extraction from text can be considered as a combination of named entity recognition (NER) and relation extraction (RE) between named entities. Challenges in chemical NER include the pervasive usage of abbreviations and aliases, deviations from standard nomenclature, and the ambiguous boundaries between which a chemical entity is defined (e.g., when multiple words describe a single species).\cite{krallinger_chemdner_2015, krallinger_information_2017} A variety of methods have been applied for chemical NER tasks. Rule-based or dictionary-based methods, such as LeadMine\cite{lowe_leadmine_2015} and ChemicalTagger\cite{hawizy_chemicaltagger_2011}, have been used to annotate reaction procedure texts or in the text parsing pipeline for constructing synthesis datasets such as SureCHEMBL\cite{papadatos_surechembl_2016}, Pistachio\cite{noauthor_nextmove_nodate}, and ZeoSyn\cite{pan_zeosyn_2024}. While these algorithms are usually computationally efficient, the scope of rules and dictionary items limits their generalizability to new datasets. Various statistical model-based NER algorithms have also been proposed, often as a sequence labeling problem where the tokens in a sentence are assigned most likely tags based on token features. A popular strategy is the use of conditional random fields\cite{lafferty_conditional_2001} in combination with expert-selected features\cite{rocktaschel_chemspot_2012} or contextualized word embeddings from neural networks (recurrent networks\cite{luo_attention-based_2018, hemati_lstmvoter_2019, zhai_improving_2019},  or transformers\cite{guo_automated_2022, isazawa_single_2022, almeida_chemical_2022,trewartha_quantifying_2022}).

    Traditionally, RE is formulated as a downstream task to NER and is solved as an ensemble of classification problems for entity pairs.\cite{hoffmann_knowledge-based_2011,riedel_modeling_2010} More recent efforts aim to solve NER and RE simultaneously by building end-to-end models.\cite{zeng_extracting_2018, miwa_end--end_2016, huguet_cabot_rebel_2021, eberts_end--end_2021} This trend has persisted as pretrained large language models (LLMs) have become more accessible. LLMs have been used for NER/RE tasks in biomedicine,\cite{luo_biogpt_2022} materials,\cite{ansari_agent-based_2023} and clinical trials,\cite{datta_autocriteria_2024} showing promise as tools for structured data extraction. For example, Dagdelen \textit{et al.} developed a training pipeline for GPT-3 to extract information from scientific texts about crystalline materials as structured JSON\cite{dagdelen_structured_2024} and Walker \textit{et al.} present an iterative scheme to fine-tune LLMs for extracting structured data of gold nanorods synthesis.\cite{walker_extracting_2023} Recent studies by Zhong \textit{et al.} explored fine-tuned LLMs for reaction data extraction from literature in PDF format.\cite{zhong-etal-2023-reactie, zhong-etal-2023-reaction} The output of these models provides a reasonable coverage of reaction information, with the exception of quantity information. Pretrained LLMs can also be used for this task directly without fine-tuning. For example, a recent preprint by Patiny and Godin explores extracting analytical experiment results from literature solely through prompt engineering.\cite{patiny_automatic_2023} While this method can extract structured data by including in-prompt data schema, it relies on closed-source LLMs and performs poorly when numerical values are involved.

    One important use case for extracting structured reaction data is the production of procedural instructions to be used for reproducing experiments. For example, Vaucher \textit{et al.} developed a transformer-based model to translate sentences of experiment procedures into action sequences.\cite{vaucher_automated_2020} While these action sequences contain detailed information for execution, their evaluations focus more on the type of action than the parameters or objects of that action. SynthReader,\cite{mehr_universal_2020} a rule-based translator developed by Mehr \textit{et al.}, converts natural language procedures to $\chi$DL, a data schema designed for chemical operations. Such a rule-based method, despite being computationally efficient, has to be expanded/modified to adapt to a different distribution, e.g., a change in writing style. Various submissions to Cheminformatics Elsevier Melbourne University (ChEMU) evaluation lab\cite{he_extended_2020, li_extended_2021, li_extended_2022} also aim to solve the NER/RE tasks including reaction/workup steps. Since these campaigns aim at evaluating individual NER/RE tasks, they do not constitute an end-to-end solution for structured data extraction into a specific output data schema.

\end{document}